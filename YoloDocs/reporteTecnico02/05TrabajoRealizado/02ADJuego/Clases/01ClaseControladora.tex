\subsubsection{Clases controladoras} \label{ClaseCtrl}
		\begin{itemize}
			\item \textbf{PrincipalMenuCtrl:} Esta clase se encarga de la funcionalidad del menú 
			principal. Esta clase esta a cargo de:
			\begin{itemize}
				\item Empezar nueva partida.
				\item Cargar nueva partida.
				\item Mostrar mensajes de confirmación antes de proceder con cambios 
				irreversibles a los datos de partida.
				\item Mostrar mensaje de aviso en caso de no encontrar exista una partida 
				guardada. 
			\end{itemize}
			%=======================================
			\item \textbf{SelectLevelMenu:} Esta clase controla la funcionalidad del menú de 
			selección nivel. Esta clase realiza:
			\begin{itemize}
				\item Habilitar solo los niveles y cinemáticas que el jugador haya 
				desbloqueado.
				\item No permitir que el jugador pueda acceder a niveles o cinemáticas 
				que el jugador no haya desbloqueado.
				\item Direccionar al jugador al nivel o a la cinemática que seleccionó.
			\end{itemize}
			%=======================================
			\item \textbf{GameDataCtrl:} Esta clase controla el archivo de los datos de 
			partida, este archivo sera de formato binario, este formato es un tipo de 
			archivo que propio de Unity y permite proteger los datos de las partidas evitando
			que estos puedan ser modificados por el jugador, garantizando así la integridad 
			de la información. Está ligada a la mayoría de los controladores pues de ella depende 
			 guardar y cargar el progreso del jugador para inicializar valores como: la 
			 vida del jugador, su cantidad de \textit{Tonalli}, los niveles disponibles, etc. 
			 Dentro de su funcionalidad está: 
			\begin{itemize}
				\item Verificar la existencia del archivo de datos de partida.
				\item Crear un archivo de datos de partida.
				\item Leer los datos del archivo de datos de partida.
				\item Escribir datos en el archivo de partida.
			\end{itemize}
			%=======================================
			\item \textbf{DialogueCtrl:} Esta clase se encarga del despliegue de diálogos
			en las cinemáticas y dentro de los niveles. Esta clase realiza las siguientes 
			actividades:
			\begin{itemize}
				\item Iniciar el despliegue de diálogos.
				\item Mostrar el dialogo siguiente.
				\item Finalizar el despliegue diálogos. 
			\end{itemize}	
			%=======================================
			\item \textbf{TalkedCharactersCtrl:} Esta clase asigna una instancia de la 
			clase \textit{Dialogue} a cada una de las instancias de la clase 
			\textit{TalkedCharacter}.
			%=======================================
			\item \textbf{CutsceneCtrl:} Esta clase se encarga de vincular el despliegue 
			de diálogos con las animaciones de las cinemáticas. Esta clase tiene diversas 
			clases hijas que heredan su funcionalidad de vinculación de diálogos y animación 
			incorporando las consideraciones necesarias para el control de cada cinemática.
			%=======================================
			\item \textbf{AudioCtrl:} Esta clase está a cargo de generar los sonidos de \textit{SFX} 
			dentro del juego utilizando la posición del jugador o de los enemigos. 
			%=======================================
			\item \textbf{LevelCtrl:} Esta clase controla el nivel que el jugador está 
			jugando. Esta clase realiza las siguientes acciones:
			\begin{itemize}
				\item Inicializar los atributos de la clase \textit{Player}.
				\item Verificar que el jugador esté vivo.
				\item Actualizar la barra de vida del jugador.
				\item Actualizar la barra de cantidad \textit{Tonalli}.
				\item Pausar el juego.
				\item Reanudar juego pausado. 
			\end{itemize}
			Esta clase tiene clases hijas que se encargan de:
				\begin{itemize}
					\item Verificar que se cumplan los objetivos específicos del nivel.
					\item Actualizar los objetivos del nivel.
					\item Actualizar los contadores de los objetivos.
					\item Guardar el progreso obtenido en el nivel.
					\item Inicializar los valores del jugador con base al \textit{checkpoint} activo. 
				\end{itemize}
			%=======================================
			\item \textbf{CameraCtrl:} Esta clase controla el desplazamiento de la cámara.
			%=======================================
			\item \textbf{MobileUICtrl:} Esta clase se encarga de comunicar al jugador 
			con la clase \textit{Player}. Es a través de esta clase que el jugador puede controlar 
			al personaje jugable. Esta clase le permite al jugador:
			\begin{itemize}
				\item Mover al personaje jugable a la derecha.
				\item Mover al personaje a la izquierda.
				\item Detener el movimiento del jugador.
				\item Actualizar la barra de cantidad \textit{Tonalli}.
				\item Pausar el juego.
				\item Reanudar juego pausado. 
			\end{itemize}
			%=======================================
			\item \textbf{ArrowCreator:} Esta clase crea objetos que instancían al prefab "Arrow".								
		\end{itemize}	