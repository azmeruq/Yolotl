\section{Teoría de autómatas}\label{teoriaAutomata}
Se recuerda que la teoría de autómatas es el estudio de dispositivos de razonamiento abstractos, es decir, de las "maquinas".

Dentro del desarrollo de videojuegos existen personajes no jugables, abreviados como NPC del inglés (no playable character) y para su funcionamiento se utilizan diferentes métodos o teorías. Para los niveles tres, cinco, siete y nueve se utiliza la teoría de autómatas. 

La información a continuación está tomada del libro "Teoría de autómatas, lenguajes y computación"\cite{libroteo}.
\\[1pt]
	
\subsubsection{Autómata finito determinista}
Dentro de los autómatas se puede encuentra la división de un autómata finito, este tiene un conjunto de estados y su control pasa de un estado a otro en respuesta a las entradas externas.

Después tenemos los autómatas deterministas y no deterministas, el primero es aquel que solo puede estar en un único estado después de cualquier secuencia de entradas, es decir, que paracada entrada al autómata existe uno y sólo un estado al que el autómata puede hacer la transición a partir de su estado actual y el no determinista permite que se pueda estar en diversos estados dada una misma entrada.

Un autómata finito constará para este trabajo:
\begin{itemize}
	\item Un conjunto finito de estados será designado como Q.
	\item Un conjunto finito de símbolos de entrada designado como \sigma.
	\item Un estado inicial designado como q0.
	\item Una función de transición designada como \delta.
	\item Un conjunto de estados finales o de aceptación desigando como F.
\end{itemize}

Por último como notación para un autómata finito determinista se utilizará un diagrama de transiciones.




\section{Listas}\label{listas}
Son estructuras lineales flexibles, yaa que pueden crecer y acortarse según se requiera, insertando o suprimiendo elementos tanto en los extremos como en cualquier otra posición de la lista.

Desde un punto de vista matemático una lista e suna secuencia de cero o más elementos de un tipo determinado.
\section{Patrones de diseño}\label{patronesDiseno}
Un patrón de diseño se caracteriza como una relación entre cierto contexto, un problema y una solución. El objetivo es proporcionar una solución que satisfaga mejor el sistemas, que se apropiada al contexto en el que se encuentra y vuelva parte del sistema para resolver un sistema mayor.

Existen diferentes clases de patrones; creacionales, estructurales y conductuales.
Aquella que nos interesa está dentro de los patrones creacionales, estos se centran en la creación, composición y representación de objetos. Los patrones creacionales ofrecen mecanismos que hacen más fácil la formación de las instancias de los objetos dentro de un sistema y establecen restricciones.

En el trabajo se usó el patrón de instancia única, que restringe la formación de instancias de una clase a un objeto.