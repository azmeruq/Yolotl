\subsection{Prueba unitaria}
Las primera prueba unitaria fue sobre los actores. En esta sección se describe
como se realiza la prueba y como se solucionan los errores encontrados a partir
de ella.

\subsubsection{Objetivo}
Verificar el funcionamiento lógico de los componentes del nivel y definir los
valores a algunos atributos para el funcionamiento correcto de algunos actores.

\subsubsection{Herramientas}
Para la realización de esta prueba se utiliza el motor gráfico de 
\textit{Unity.}

\subsubsection{Aplicación}
Para esta prueba se evalúa el comportamiento de los actores antes de su
integración a los niveles y cómo estos interactúan con el jugador. Unity permite
ver los valores que adquieren los atributos durante su ejecución, ver figura
\ref{fig:Debug01}. Así que para esta prueba basta con ejecutar la escena
base y observar cómo responden los actores.
\\
\par
Primeramente se revisa que los enemigos y las plataformas siguen sus patrones de
movimiento definidos. Después se verifica que los enemigos, obstáculos e ítems
afecten la cantidad de vida del jugador y en algunos casos su cantidad de
\textit{tonalli}. En el caso de los jefes se verifica que sus ataques no se vean
interrumpidos por nuevos ataques de la máquina de estados. En el caso particular
de obstáculos como \textit{WindCreator} se definen los intervalos de tiempo
para su correcto funcionamiento.

        \begin{figure}[h]
                \centering
                \includegraphics[width=0.2\textwidth]{04ResultadosObetnidos/imagenes/enemyPruebas01.png}
                \caption{Unity permite ver los valores de los tributos de las clases en ejecución.}
                \label{fig:Debug01}
        \end{figure}
        
\subsubsection{Resultados}
En esta prueba se observan diferentes problemas en el comportamiento de los actores
que se solucionan, a continuación se mencionan los errores encontrados y como se
solucionaron:
        \begin{itemize}
                \item El marcador se actualiza al doble cuando el jugador cae
                sobre un objeto coleccionable: Este error resulta producto de
                utilizar un \textit{GameObject} auxiliar para la detección de las
                colisiones del suelo. El error se soluciona fácilmente al agregar
                un componente de tipo rigidbody 2D al \textit{GameObject} auxiliar
                para la detección de las colisiones del suelo.
                \item Los ítems restauran el doble de vida cuando el jugador cae
                sobre ellos: Este error es generado por las mismas causas que el
                de los objetos coleccionables así que al solucionar el de los
                objetos coleccionables se soluciona éste.
                \item Los ataque de los jefes generados por corrutinas se
                empalman con otros ataques o interrumpen los que ya se están
                ejecutando: Esto se soluciona al detener todas la corrutinas generadas
                por el jefe cuando se ejecuta un ataque.
        \end{itemize}

\subsubsection{Conclusiones}
El diseño de los componentes del juego es el indicado si se desea evaluar 
componentes individuales antes de un integración completa. Esta evaluación 
de componentes agiliza la detección de errores y su corrección.