	\subsubsection{Metodología en cascada}
La metodología de desarrollo en cascada o también conocida como modelo de vida lineal o básico,  fue propuesta por Royce en 1970 y a partir de entonces ha tenido diferentes modificaciones. Sigue una progresión lineal por lo que cualquier error que no se haya detectado con antelación afectara todas las fases que le sigan provocando una redefinición en el proyecto y por ende un aumento en los costos de producción del sistema [ ].
Esta metodología se divide en las siguientes etapas:
\begin{itemize}
	\item \textbf{Análisis de los requisitos del software}: En esta etapa se recopilan los requisitos del sistema, se centra especialmente en toda aquella información que pueda resultar de utilidad en la etapa de diseño, tales como tipos de usuarios del sistema, reglas de negocio de la empresa, procesos, etc. En esta etapa se responde la pregunta de ¿Qué se hará? 
	\item \textbf{Diseño}: Esta etapa se caracteriza por definir todas aquellas características que le darán identidad al sistema, tales como la interfaz gráfica, la base de datos, etc. Las características anteriormente definidas se obtendrán de la etapa de análisis. En esta etapa se respondería la pregunta de ¿Cómo se hará? 
	\item \textbf{Codificación}: Terminada la etapa de diseño, lo siguiente es programar y crear todos los elementos necesarios para el funcionamiento del sistema. 
	\item \textbf{Prueba}: Finalizada la decodificación se debe de probar la calidad del sistema. En este punto es importante resaltar que la pruebas no solo abarcan que se confirme que el sistema funcione, sino que también verifica que los usuarios puedan aprender a utilizarlo con facilidad, entre otros aspectos como la seguridad de la información y los tiempos de respuesta del sistema.
	\item \textbf{Mantenimiento}: En esta última etapa se realizarán modificaciones al sistema, sin que esto necesariamente signifique que estos cambios se deban a errores de programación, puesto que esta etapa también abarca agregar nueva funcionalidad al sistema o, en caso de que trabaje con protocolos de estándar internacional, actualizar sus protocolos \cite{Ref:CarCascada}. 
\end{itemize}
Algunos de los inconvenientes que presenta son:
\begin{itemize}
	\item No refleja el proceso de desarrollo real.
	\item Tiempos largos de desarrollo.
	\item Poca comunicación con el cliente.
	\item Revisiones de proyecto de gran complejidad.
\end{itemize}