\section{Videojuego}
\subsection{Definición}
Un videojuego es un medio de entretenimiento que involucra a un usuario, denominado jugador, en una interacción constante entre una interfaz y un dispositivo de video\cite[Morales Urrutia, 2010]{defVid}. Los videojuegos recrean	entornos y situaciones virtuales en los que el jugador puede controlar la situación para alcanzar objetivos por medio de determinadas reglas. La interacción se lleva a cabo mediante dispositivos de salida y de entrada.
\\[1pt]
		
Los videojuegos arte, ciencia y tecnología; involucran una plétora de habilidades y conocimientos en distintas disciplinas, desde ciencias formales hasta ciencias sociales que van más allá del típico proyecto de software e implican al mismo tiempo la creatividad y la imaginación.
\\[1pt]
					
\subsection{Clasificación}
Categorizar cada pieza individual y reunir en grupos que comparten características similares con fines de organización y facilitar la comunicación respecto de un tema.
\\[1pt]

El documento \cite[MDA]{vid10} (por sus siglas en inglés) establece los tres aspectos fundamentales, Mecánicas, Dinámicas y Estética como un marco de referencia para entender los juegos y hacer una mejor clasificación de ellos. Las mecánicas, son las reglas y sistemas que crean nuestra experiencia de juego, sus componentes particulares a nivel representación de datos y algoritmos; las dinámicas, describen el comportamiento de las mecánicas en respuesta a las acciones del jugador; y la estética, define la respuesta emocional evocada por el usuario cuando interactúa con el juego. 
\\[1pt]

Mark Wolf\cite{vid11} enfatiza la diferencia fundamental de los videojuegos con otros medios y el por qué de ser necesaria una categorización en géneros muy distinta a la aplicable a libros y películas. La participación del jugador es el determinante central a la hora de describir y clasificar juegos de video. Mark Wolk comenta que "Por supuesto, cualquier sistema propuesto será objeto de debate y crítica. Al mismo tiempo, aparecer con un listado consistente y comprensivo que intenta definir y articular las fronteras de cada uno, es una tarea mucho más difícil que criticar otros ya existentes".
\\[1pt]

Sin embargo, se abarcarán las clasificaciones determinadas por el mercado de la industria. La clasificación por contenido son definidas por asuntos legales que competen a cada región. La clasificación por géneros son las razones emotivas o de experiencia que tenemos para consumir.
\\[1pt]

\subsubsection{Clasificación por contenido}
	Es usado para la clasificación de videojuegos en grupos idóneos relacionados por su contenido. Existen diferentes sistemas en el mundo donde la mayoría de estos están asociados y patrocinados por un gobierno y a veces forman parte del sistema de clasificación de películas del país.\\[1pt]	
	
	México pertenece a las clasificaciones de la Junta de Clasificación de Software de Entretenimiento (ESRB, Entertainment Software Rating Board)\cite{vid02}. Esta clasificación proporciona una información concisa y objetiva acerca del contenido de los juegos de video y las aplicaciones para que los consumidores, en especial los padres, puedan tomar decisiones informadas. Las clasificaciones de la ESRB constan de tres partes:\\[1pt]
			
\textbf{Por edad: }
Sugieren la edad adecuada para el juego.		
\\[1pt]
	
			\begin{itemize}
			\item Niños pequeños
			El contenido está dirigido a niños pequeños.
			
			\item Todos
			El contenido por lo general es apto para todas las edades. Puede que contenga una cantidad mínima de violencia de caricatura, de fantasía o ligera, o uso poco frecuente de lenguaje moderado.
			
			\item Todos +10
			El contenido por lo general es apto para personas de 10 años o más. Puede que contenga más violencia de caricatura, de fantasía o ligera, lenguaje moderado o temas mínimamente provocativos.
			
			\item Adolescentes
			El contenido por lo general es apto para personas de 13 años o más. Puede que contenga violencia, temas insinuantes, humor grosero, mínima cantidad de sangre, apuestas simuladas o uso poco frecuente de lenguaje fuerte.
			
			\item Maduro
			El contenido por lo general es apto para personas de 17 años o más. Puede que contenga violencia intensa, derramamiento de sangre, contenido sexual o lenguaje fuerte.
			
			\item Adultos únicamente
			El contenido es apto sólo para adultos de 18 años o más. Puede que incluya escenas prolongadas de violencia intensa, contenido sexual gráfico o apuestas con moneda real.
			
			\item Clasificación pendiente
			Aparece solo en material de publicidad, de comercialización y promocional en relación con un videojuego "en caja" que se espera que lleve una clasificación de la ESRB y debe reemplazarse por la clasificación del juego una vez que haya sido asignada.
			
		\end{itemize}		
			
			\textbf{Descriptores de contenido: } 
			Indican los elementos que pueden haber motivado la clasificación asignada y pueden resultar de interés o preocupación.
			\\[1pt]
			
			\begin{itemize}
				\item Referencia al alcohol: referencia e imágenes de bebidas alcohólicas.
				\item Animación de sangre: representaciones decoloradas o no realistas de sangre.
				\item Sangre: representaciones de sangre.
				\item Derramamiento de sangre: representaciones de sangre o mutilación de partes del cuerpo.
				\item Violencia de caricatura: acciones violentas que incluyen situaciones y personajes caricaturescos. Puede incluir violencia en la cual un personaje sale ileso después de que la acción se llevó a cabo.
				\item Travesuras cómicas: representaciones o diálogo que impliquen payasadas o humor sugestivo.
				\item Humor vulgar: representaciones o diálogo que implique bromas vulgares, incluido el humor tipo “baño”.
				\item Referencia a drogas: referencia o imágenes de drogas.
				\item Violencia de fantasía: acciones violentas de naturaleza fantástica que incluyen personajes humanos y no humanos en situaciones que se distinguen con facilidad de la vida real.
				\item Violencia intensa: representaciones gráficas y de apariencia realista de conflictos físicos. Puede comprender sangre excesiva o realista, derramamiento de sangre, armas y representaciones de lesiones humanas y muerte.
				\item Lenguaje: uso de lenguaje soez de moderado a intermedio.
				\item Letra de canciones: referencias moderadas de lenguaje soez, sexualidad, violencia, alcohol o uso de drogas en la música.
				\item Humor para adultos: representaciones o diálogo que contienen humor para adultos, incluidas las alusiones sexuales.
				Desnudez: representaciones gráficas o prolongadas de desnudez.
				\item Desnudez parcial: Representaciones breves o moderadas de desnudez.
				\item Apuestas reales: el jugador puede apostar, incluso colocar apuestas con dinero o divisas de verdad.
				\item Contenido sexual: representaciones no explícitas de comportamiento sexual, tal vez con desnudez parcial.
				\item Temas sexuales: alusiones al sexo o a la sexualidad.
				\item Violencia sexual: representaciones de violaciones o de otros actos sexuales violentos.
				\item Apuestas simuladas: el jugador puede apostar sin colocar apuestas con dinero o divisas reales.
				\item Lenguaje fuerte: uso explícito o frecuente de lenguaje soez.
				Letra de canciones fuerte: alusiones explícitas o frecuentes de lenguaje soez, sexo, violencia o uso de alcohol o drogas en la música.
				\item Contenido sexual fuerte: alusiones explícitas o frecuentes de comportamiento sexual, tal vez con desnudez.
				\item Temas insinuantes: referencias o materiales provocativos moderados.
				\item Referencia al tabaco: referencia o imágenes de productos de tabaco.
				\item Uso de alcohol: consumo de alcohol o bebidas alcohólicas.
				\item Uso de drogas: consumo o uso de drogas.
				\item Uso de tabaco: consumo o uso de productos de tabaco.
				\item Violencia: escenas que comprenden un conflicto agresivo. Pueden contener desmembramiento sin sangre.
				Referencias violentas: alusiones a actos violentos.
			\end{itemize}
			
			\textbf{Elementos interactivos: } 
			Informan acerca de los aspectos interactivos de los productos, incluida la capacidad de los usuarios de interactuar, o si se comparte la ubicación de los usuarios con otros usuarios.
			\\[1pt]
			
			\begin{itemize}
			
			\item Ubicación compartida: Incluye la capacidad de mostrar la ubicación del usuario a otros usuarios de la aplicación.
			\item Interacción de usuarios: Indica una posible exposición a contenido sin filtro y sin censura generado por usuarios, que incluye comunicaciones y medios compartidos de usuario a usuario a través de medios y redes sociales.
			\item Compras digitales: Permite la compra de productos digitales directamente desde la aplicación.
			\item Internet sin límites: El producto brinda acceso a Internet.
		\end{itemize}
		
	\subsubsection{Clasificación por género}
	A lo largo de la historia de los videojuegos, sus creadores han ido dando lugar a una variedad creciente de géneros en las distintas plataformas disponibles. Estos géneros se han ido conformando en torno a factores como: la representación gráfica, el tipo de interacción entre el jugador y la máquina, la ambientación, y su sistema de juego, siendo este último el criterio más habitual a tener en cuenta. Por lo dicho anteriormente de la clasificación, existen diferentes divisiones y subdivisiones por varios autores. A continuación se presenta una de ellas en la imagen \ref{fig:vidGen}.
	\\[1pt]
	
	\begin{figure}
		\centering
		\includegraphics[width=\textwidth]{03MarcoTeorico/imageR/gene.png}
		\caption{Géneros de videojuegos propuesto por Luis Chong \cite{vid12}.}
		\label{fig:vidGen}
	\end{figure}
	
	
		
\subsection{Industria mundial}
			 
			 El videojuego surge en 1952; no obstante, el videojuego como industria surgiría hasta 1972, logrando su mayor revolución durante la década de los 80`s. 
			 Según un estudio elaborado por la empresa Newzoo, la industria del videojuego generará 108.900 millones de dólares de ingresos totales, de los que se espera que hasta 94.400 millones corresponden solamente a ventas digitales, que representa un 87\% del mercado mundial. Actualmente la industria del videojuego, también llamada industria del ocio virtual, es la industria del entretenimiento, superando a la industria del cine y la música.
			 \\[1pt]	
			 
			 El segmento de los dispositivos móviles (Smartphones y tablets) es el que aporta más dinero en la industria del videojuego. Este sector copa un 42\% del mercado y su consumo ha tenido un crecimiento del 19\% con respecto al año anterior. Se espera que generen un ingreso de 46.100 millones de dólares. A día de hoy no se entiende a ninguna persona sin su Smartphone en la mano y esto hace que un gran porcentaje de usuarios juegue a algún tipo de juego en su teléfono y hasta utilice navegadores de internet sin necesidad de instalar nada. Se espera que en 2020 acaparen el 50\% del mercado. \cite{vid01}
			 \\[1pt]
			 
\subsection{Estudio de mercado en México}

Históricamente, México ha sido el país numero uno en el consumo de videojuegos en Latinoamérica. Esto se debe a su cercanía con los Estados Unidos. Esto genera que se de una transmisión cultural y de tecnología casi inmediata. La industria de los videojuegos en México es cosa seria. Mientras que la economía nacional creció sólo 2.2\% en 2016 y se prevé que lo haga entre 1.3 y 2.3\% en este año, este mercado de entretenimiento pronostica un crecimiento anual de 8.4\% para 2017, es decir, casi cuatro veces lo que creció el Producto Interno Bruto (PIB) hace un año y lo que avanzaría al cierre del periodo en curso. De acuerdo con el estudio “Jugar no es cosa de niños: Dimensionamiento del Mercado de Videojuegos en México 1Q17”, elaborado por The Competitive Intelligence Unit (The CIU), este mercado tuvo ingresos por más de 22,852 millones pesos (mdp) en 2016, esto es, 13.3\% más con respecto al año anterior, con un número de usuarios de más de 65 millones \cite{vid03}. De esos casi 23 mdp (que duplica los 11,278 mdp registrado en 2009), el 58\% de los ingresos, es decir, 13,347 mdp, proviene del software (videojuegos, apps, etcétera), y el restante 42\% (9,514 mdp) se obtiene del hardware (consolas y dispositivos portátiles). Se estima que el mercado de videojuegos alcanzará los 24,771 y 27,032 mdp en 2017 y 2018, confirmando su dinamismo ante la masificación de los dispositivos móviles y el desarrollo de nuevas tecnologías que permiten nuevas modalidades y capacidades de juego. El estudio revela que 63.7\% de los encuestados se asume como un usuario frecuente, que juega entre 1 y 3-4 veces a la semana; aunque de ese porcentaje el 40\% se asegura que juega entre una y dos veces por semana. Mientras que los ocasionales representan la parte menor mientras que los ocasionales representan la parte menor con 5.4\%. De los llamado intensivos, por su parte, el 31.5\% juega diario o 5-6 veces a la semana.


\subsection{Industria en México}
La industria de producción de videojuegos en México se encuentra actualmente en una fase de desarrollo, debido a la persistente falta de oportunidades para desarrollarse en este tipo de actividad bajo un esquema corporativo o empresarial. Lo anterior se ve reflejado en la distribución del tipo de empleo de los desarrolladores nacionales, puesto que existe una alta proporción de empleados dedicados a la creación de videojuegos bajo un esquema independiente, son pocos casos los que llegan a consolidar su creación en una empresa con generación de empleos e ingresos en el largo plazo.
\\[1pt]

En México, la mayoría de empresas son micropymes, y no existe información abierta sobre su facturación o cuantas de ellas todavía no facturan. Muchas de estas pequeñas empresas recurren a soluciones como el crowndfounding mediante plataformas como Kickstarter para financiar su proyecto y buscan mentoring en las comunidades de desarrolladores cercanas\cite{vid05}. De acuerdo a estudios recientes, 40\% de los desarrolladores de videojuegos en México trabajan de modo independiente, mientras que únicamente 10\% de los desarrolladores han consolidado su propio negocio. Esto demuestra que una gran proporción de esta mano de obra se encuentra deslindada de grandes corporativos. En el caso de nuestro país, 6 de cada 10 desarrolladores dedican su actividad al desarrollo en smartphones y 32\% en tabletas, mientras que únicamente 26\% se especializan en el desarrollo de juegos en consolas fijas, respondiendo a una demanda de 40.7 millones de mexicanos que utilizan sus smartphones como principal dispositivo de juego \cite{vid04}.
\\[1pt]

Lista de estudios activos:
\begin{itemize}
	\item Larva Game Studios
	\item Kaxan Games
	\item Xibalba Studios
	\item Estudios Maquina Voladora
	\item Slang Studio
	\item Golden Pie Studio
	\item Kokonut Studio
	\item Phyne Games
	\item Playful Studios
	\item Squad Games
	\item Washa Washa
	\item Hollow Games
	\item HyperBeard Games
	
\end{itemize}
