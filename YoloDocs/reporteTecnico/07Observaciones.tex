\chapter{Conclusiones}

% punto final del informe: "como ha sido mostrado...."

Como se vió durante el proyecto, un videojuego educativo es una gran herramienta de ayuda para la enseñanza y el aprendizaje. Este es un medio controlado con normas establecidas, lo que facilita en control de información y en que modo debe aprenderse.

Se demostró que un videojuego en la actualidad es grandemente aceptado por la sociedad, tanto en jóvenes como en adultos. Y aquellos adultos con familia están dispuestos a realizar gastos en compra de este tipo de productos. 

También se vió que la cultura puede ser vista entretenida gracias al juego. Se puede combinar de manera digerible por el jugador los componentes históricos y de juego. Y este tipo de combinación en material cultural con un videjuego es bien recibido por casi todos los diferentes tipos de personas.

Al final ha sido demostrado, que con los incentivos correctos para grupos específicos de personas, en este caso un videojuego, se puede motivar a la sociedad a que se interese y aprenda conocimiento cultural. 