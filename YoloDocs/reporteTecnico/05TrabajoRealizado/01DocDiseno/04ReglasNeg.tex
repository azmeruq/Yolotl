\subsection{Reglas de negocio.}
A partir del documento de diseño se identificaron las siguientes reglas de negocio para el desarrollo del juego. 

\begin{longtable}[c]{ | m{5cm} | m{10cm}|} 
		\hline
		\rowcolor{cyan}Regla de Negocio & Descripción \\ 
		\hline
		%===================================
		RN-001\label{RN:01} & El juego solo permitirá una partida. \\ 
		\hline
		%===================================
		RN-002\label{RN:02} & La progresión dentro del juego será lineal. Es decir, el nivel n solo se desbloqueará si se ha completado el nivel n-1. Siendo el nivel 1 el que estará disponible por default.\\ 
		\hline
		%===================================
		RN-003\label{RN:03} & El archivo de datos de la partida usara un tipo de cifrado para evitar que el jugador lo pueda modificar.\\ 
		\hline
		%===================================
		RN-004\label{RN:04} & El personaje jugable durante los niveles 1 al 10 es el personaje de Malinalli. Sólo en el nivel 9, contará con otro personaje jugable además de Malinalli: Xólotl. \\ 
		\hline
		%===================================
		RN-005\label{RN:05} & Los niveles del juego ubicados en el Mictlán estarán divididos en dos etapas: una de plataformas y otra donde se deberá eliminar al guardián del nivel, salvo por el nivel 10. \\ 
		\hline
		%===================================
		RN-006\label{RN:06} & Existirán dos tipos de enemigos. Enemigos tipo normal y enemigos tipo Jefe. \\ 
		\hline
		%===================================
		RN-007\label{RN:07} & La información contenida en los checkpoints se mantendrá activa mientras el jugador se mantenga dentro del nivel. Una vez que el jugador sale del nivel esta información se elimina.\\ 
		\hline
		%===================================
		RN-008\label{RN:08} & La resolución que se manejará en los sprites será de 70 ppx.\\ 
		\hline
		%===================================
		RN-009\label{RN:09} & Los sprites solo contendrán colores en 8 bits.\\ 
		\hline
		%===================================
		RN-010\label{RN:10} & El jugador solo podrá dialogar con aquellos personajes que tengan un icono de dialogo. \\ 
		\hline
		%===================================
		RN-011\label{RN:11} & Los cuadros de diálogos solo contendrán  “” caracteres. En caso de que el dialogo exceda esa cantidad, el mensaje se dividirá en diferentes cuadros de diálogos hasta que se haya mostrado todo el mensaje. \\ 
		\hline
		%===================================
		RN-012\label{RN:12} & Cuando el jugador este dialogando con un personaje, la funcionalidad de los demás botones de la GUI se deshabilitarán.\\ 
		\hline
		%===================================
		RN-013\label{RN:13} & El jugador cuenta con una cantidad de tonalli determinada en cada disparo se gasta una cantidad por disparo.\\ 
		\hline
		%===================================
		RN-014\label{RN:14} & Cuando la barra de tonalli llega a cero, se restaurará después de 15 segundos.\\ 
		\hline
		%===================================
		RN-015\label{RN:15} & La cantidad de tonalli se puede restaurar tocando el ítem flor de vainilla.\\ 
		\hline
		%===================================
		RN-016\label{RN:16} & La habilidad de disparar tonalli se encuentra desbloquea al finalizar el nivel 1.\\ 
		\hline
		%===================================
		RN-017\label{RN:17} & El jugador no puede efectuar más de dos saltos de manera consecutiva.\\ 
		\hline
\end{longtable}