\chapter{Introducción}

En la época actual nos vemos rodeados por tecnología por todos lados, esta ya es parte de nuestra vida diaria, de nuestras actividades tanto de trabajo, de nuestro medio de comunicación, como medio de información e incluso de nuestro medio de entretenimiento. Las generaciones recientes han crecido con la evolución de la tecnología a un ritmo acelerado, a tal punto que la tecnología ya es parte de su cultura.

Dentro de la evolución de la tecnología se encuentra los videojuegos, una industria de entretenimiento.  
Nos daremos a la tarea de investigar los puntos de impacto que tiene. Pues a primera instancia podremos observar el tipo de personas que juegan, los ingresos que se generan en esta industria, los tipos de industria que existen, las consecuencias positivas, las consecuencias negativas que generarían, las ganancias como profesionista en esta rama y como realizar un proyecto de esta naturaleza.

Los videojuegos hacen al jugador involucrarse con varios sentidos a la vez en lo que se le presenta, así se crea una experiencia propia como cualquiera de la vida pues provoca la inmersión. El INJUVE ha dado a conocer información donde se puede observar quelos jóvenes y los videojuegos llevan una interacción diaria y sobre cualquier tema, desde educativo hasta de ocio.

Por otra parte tenemos que la sociedad ignora en su mayoría los aspectos históricos culturales propios de su país, específicamente de México.Se denota un gran desinterés por parte de la gente el siquiera conocer su legado.

Tomando en cuenta todos los aspectos anteriores, el proyecto consistirá en juntar ambas ideas para usar el videojuego como difusión del aspecto cultural mexicano. Se investigará más a fondo la historia de los videojuegos para elegir el público onjetivo potencial o las personas alcanzables, los procesos y metodología para poder realizar un videojuego, aspectos a considerar para el proyecto, la cultura y ramas que abarca dentro de la sociedad, la cultura y la tecnología como se relacionan entre sí, las herramientas con las que se cuentan para la realización del videojuego, las teorías que pueden usarse, los conflictos que pueden ocurrir, como solucionar los problemas que se nos presenten y por su puesto el cambio o impacto que se tenga al presentarlo al público.

Al final se presentará un videojuego como producto con pruebas de diferentes tipos para demostrar el resultado ante la sociedad y los cambios que se presentaron en las personas.