\chapter{Resultados obtenidos}
En este capitulo se habla de los resultados obtenidos durante trabajo terminal 2. 
Por tal motivo en este capitulo se abordan las pruebas realizadas y las 
caracteristicas que tienen los niveles para ser considerados como acabados.

\section{Pruebas}
En esta sección se reportan todos los tipos de pruebas a los que se sometió el 
juego para probar tanto su funcionabilidad como su desempeño y el impacto que 
tuvo en los jugadores.

%%\begin{table}[prueba01]
%	\centering
%	\caption{My caption}
%	\label{my-label}
%	\begin{tabular}{ll}
%		\multicolumn{2}{l}{Prueba 01} \\
%		Objetivo                  &  asdasdasd \\
%		Herramientas utilizadas   &  asdasdasd \\
%		Aplicación                &  asdasdasd \\
%		Conclusiones              &  adasdasd
%	\end{tabular}
%\end{table}
\subsection{Prueba unitaria}
Las primera prueba unitaria fue sobre los actores. En esta sección se describe
como se realiza la prueba y como se solucionan los errores encontrados a partir
de ella.

\subsubsection{Objetivo}
Verificar el funcionamiento lógico de los componentes del nivel y definir los
valores a algunos atributos para el funcionamiento correcto de algunos actores.

\subsubsection{Herramientas}
Para la realización de esta prueba se utiliza el motor gráfico de 
\textit{Unity.}

\subsubsection{Aplicación}
Para esta prueba se evalúa el comportamiento de los actores antes de su
integración a los niveles y cómo estos interactúan con el jugador. Unity permite
ver los valores que adquieren los atributos durante su ejecución, ver figura
\ref{fig:Debug01}. Así que para esta prueba basta con ejecutar la escena
base y observar cómo responden los actores.
\\
\par
Primeramente se revisa que los enemigos y las plataformas siguen sus patrones de
movimiento definidos. Después se verifica que los enemigos, obstáculos e ítems
afecten la cantidad de vida del jugador y en algunos casos su cantidad de
\textit{tonalli}. En el caso de los jefes se verifica que sus ataques no se vean
interrumpidos por nuevos ataques de la máquina de estados. En el caso particular
de obstáculos como \textit{WindCreator} se definen los intervalos de tiempo
para su correcto funcionamiento.

        \begin{figure}[h]
                \centering
                \includegraphics[width=0.2\textwidth]{04ResultadosObetnidos/imagenes/enemyPruebas01.png}
                \caption{Unity permite ver los valores de los tributos de las clases en ejecución.}
                \label{fig:Debug01}
        \end{figure}
        
\subsubsection{Resultados}
En esta prueba se observan diferentes problemas en el comportamiento de los actores
que se solucionan, a continuación se mencionan los errores encontrados y como se
solucionaron:
        \begin{itemize}
                \item El marcador se actualiza al doble cuando el jugador cae
                sobre un objeto coleccionable: Este error resulta producto de
                utilizar un \textit{GameObject} auxiliar para la detección de las
                colisiones del suelo. El error se soluciona fácilmente al agregar
                un componente de tipo rigidbody 2D al \textit{GameObject} auxiliar
                para la detección de las colisiones del suelo.
                \item Los ítems restauran el doble de vida cuando el jugador cae
                sobre ellos: Este error es generado por las mismas causas que el
                de los objetos coleccionables así que al solucionar el de los
                objetos coleccionables se soluciona éste.
                \item Los ataque de los jefes generados por corrutinas se
                empalman con otros ataques o interrumpen los que ya se están
                ejecutando: Esto se soluciona al detener todas la corrutinas generadas
                por el jefe cuando se ejecuta un ataque.
        \end{itemize}

\subsubsection{Conclusiones}
El diseño de los componentes del juego es el indicado si se desea evaluar 
componentes individuales antes de un integración completa. Esta evaluación 
de componentes agiliza la detección de errores y su corrección.
\subsection{Prueba de integración}
Esta prueba se realiza una vez se integraron los actores y controladores a los
niveles.

\subsubsection{Objetivo}
Verificar el funcionamiento lógico de los componentes del nivel al ser integrados
para formar un nivel entero.

\subsubsection{Herramientas}
Para la realización de esta prueba se utiliza el motor gráfico de \textit{Unity.}

\subsubsection{Aplicación}
Para realizar esta prueba es necesario jugar los niveles y observar que el
comportamiento de los controladores y los actores se ejecute correctamente al
integrarse con otros actores. En esta prueba también se ajustan las áreas activas de las
plataformas a fin de que su funcionamiento no se detenga si se alejan mucho del
jugador al realizar su recorrido.
                
\subsubsection{Resultados}
Al finalizar esta prueba se puede verificar que los controladores funcionan de
manera correcta; sin embargo, es necesario realizar ajustes referentes a los
tiempos de transiciones entre escenas y los valores de las áreas activas de
varias plataformas y obstáculos ya que con sus valores iniciales algunas
plataformas se detenían al realizar su recorrido dado que el jugador se salía
de su área activa y se volvía inalcanzable. En cuanto al obstáculo de
\textit{WindCreator} se ajusto el tamaño del área activa garantizando que el
obstáculo se encuentre activo cuando el jugador llegue a donde se encuentra éste. 

\subsubsection{Conclusiones}
Al finalizar esta prueba se puede concluir lo siguiente:
\begin{itemize}
    \item El desarrollo orientado a componentes facilita identificar los errores en
    las clases actoras antes de ser integradas a los niveles.
    \item Existen errores que sólo pueden ser detectados al integrar más de un actor
    y en situaciones muy específicas como el caso de los marcadores.
    \item No solo los errores de funcionamiento en el código de las clases actoras
    impactan negativamente en la experiencia del jugador; en ocasiones la experiencia
    se ve afectada por los valores que se le asignan a las clases que componen los
    actores como es el caso de los jefes y de los obstáculos.
\end{itemize} 
\subsection{Prueba de sistema}
Esta prueba se realiza una vez se integraron los actores y controladores a los 
niveles.
\subsubsection{Objetivo de la prueba}
Verificar el flujo de la navegación del juego. 
\subsubsection{Herramientas utilizadas durante la prueba}
\textit{Unity.}
\subsubsection{Aplicación de la prueba}
Esta prueba inicia desde la escena de menú principal en donde se verifica que 
el controlador del menú realiza las validaciones correspondientes antes de 
crear o cargar una partida; de igual forma de verifica que aparezcan los 
mensajes de confirmación a cada caso, sea el de confirmación de la nueva partida 
o el que notifica que no hay datos previamente guardados.
\\
\par
La siguiente escena a probar es el menú de selección de nivel. En este se verifica 
que la información mostrada por la interfaz corresponda al nivel que se desea 
acceder. Después, se verifica que en efecto el juego no permita acceder a niveles 
que aun no se desbloquean.
\\
\par
Para finalizar la prueba se verifica que se realicen las transiciones entre 
niveles y cinemáticas. De igual forma se prueba la funcionalidad de botones de 
navegación de los niveles referentes al panel de pausa, fin de partida y nivel 
completado.
\subsubsection{Conclusiones de la prueba}
Al finalizar esta prueba se pudo confirmar que las transiciones entre escenas se 
realiza de manera correcta; salvo en algunos casos pero fue debido a que el nombre 
de la escena a la que se debería redirigir no estaba escrito correctamente o no 
coincidía con el nombre de la escena a la que debía ir. Con esto se puede concluir 
que se cumple el mapa de navegación que se propuso en el documento de diseño 
realizado en trabajo terminal 1.
\subsection{Prueba de rendimiento en \textit{Unity}}
Esta prueba se realiza una vez que hechas las modificaciones como producto de las pruebas unitarias, de integración y de sistema.
\subsubsection{Objetivo de la prueba}
Verificar el uso del GPU.
\subsubsection{Herramientas utilizadas durante la prueba}
\textit{Profiler de Unity.}
\subsubsection{Aplicación de la prueba}
Esta prueba inicia desde el menú principal y con la herramienta \textit{profiler} se
observa el desempeño del GPU de la máquina al simular el juego. La ventaja de utilizar
\textit{Profiler} es que indica que elementos de la escena son los que están
consumiendo un determinado porcentaje del GPU. En las figuras
\ref{fig:MenuSelectionprofiler}, \ref{fig:CutSeceneprofiler} y
\ref{fig:LevelProfiler} se muestran los resultados de la
herramienta \textit{Profiler}.
\begin{figure}
  \centering
 
   \subfigure[Vista general del uso del GPU.] {\includegraphics[width=0.6 \textwidth]
   {04ResultadosObetnidos/imagenes/ProfilerMenuselector01.png}}
   
        \subfigure[Desglose de los porcentajes del uso del GPU] {\includegraphics[width=0.6 \textwidth]{04ResultadosObetnidos/imagenes/ProfilerMenuselector02.png}}
        
        \subfigure[Desglose de los porcentajes del uso de memoria] {\includegraphics[width=0.6 \textwidth]{04ResultadosObetnidos/imagenes/ProfilerMenuselector04.png}}
  \caption{Resultados de la herramienta \textit{profiler} al analizar el menú de        
  seleccion.}
  \label{fig:MenuSelectionprofiler}
\end{figure}

\begin{figure}
  \centering
 
   \subfigure[Vista general del uso del GPU.] {\includegraphics[width=0.6 \textwidth]
   {04ResultadosObetnidos/imagenes/cutsceneProfiler01.png}}
        
        \subfigure[Desglose de los porcentajes del uso de memoria] {\includegraphics[width=0.3 \textwidth]{04ResultadosObetnidos/imagenes/cutsceneProfiler02.png}}
  \caption{Resultados de la herramienta \textit{profiler} al analizar una cinemática.}
  \label{fig:CutSeceneprofiler}
\end{figure}

\begin{figure}
  \centering
 
   \subfigure[Vista general del uso del GPU.] {\includegraphics[width=0.6 \textwidth]
   {04ResultadosObetnidos/imagenes/Level02Profiler03.png}}
        
        \subfigure[Desglose de los porcentajes del uso del GPU] {\includegraphics[width=0.6 \textwidth]{04ResultadosObetnidos/imagenes/Level02Profiler04.png}}  
        
        \subfigure[Desglose de los porcentajes del uso de memoria] {\includegraphics[width=0.6 \textwidth]{04ResultadosObetnidos/imagenes/Level02Profiler08.png}}
  \caption{Resultados de la herramienta \textit{profiler} al analizar un nivel.}
  \label{fig:LevelProfiler}
\end{figure}
\subsubsection{Conclusiones de la prueba}
Al observar el desglose del uso del GPU en las diferentes escenas que se
probaron, se identifica al \textit{EditorOverHead} como uno de los principales
consumidores de recursos; investigando en la documentación de \textit{Unity},
se detecta que este elemento es producto de un error de rendimiento en la
versión 2017 pero que se puede solucionar al descargar uno de los parches que
\textit{Unity} proporciona desde su sitio web.
\\
\par
Con las pruebas de \textit{profiler} se puede concluir que el juego tiene un buen
rendimiento en cuanto a uso de recursos puesto que no presenta caídas
dramáticas en cuanto a desempeño.
\subsection{Prueba de rendimiento en teléfono \textit{Huawei TAG-L13}}
Esta prueba se realiza una vez que hechas las modificaciones como producto de las pruebas unitarias, de integración y de sistema.
\subsubsection{Objetivo de la prueba}
Verificar el uso del GPU y el nivel de batería del teléfono que utiliza mientras la aplicación esté funcionando.
\subsubsection{Herramientas utilizadas durante la prueba}
Opciones de desarrollador del teléfono Huawei TAG-L13 y \textit{Battery Doctor}.
\subsubsection{Aplicación de la prueba}
Para esta prueba se debe de instalar la apk del juego en el dispositivo, activar las
opciones de desarrollador e instalar la aplicación \textit{Battery Doctor}.  Una vez
hecho esto se juega el juego y se mide el desempeño desde el teléfono. En las figura
\ref{fig:GPUHuawei} se muestra el uso del GPU en distintos momentos de la
partida. Por otra parte en la figura \ref{fig:BateriaYolotl} se muestra el uso
de la batería y el uso promedio de \textit{GPU} que mide \textit{Battery Doctor}.
\\
\par
\begin{figure}
  \centering
 
   \subfigure[Uso del GPU desde el menú principal.] {\includegraphics[width=0.4 \textwidth]{04ResultadosObetnidos/imagenes/rendimiento01.png}}
   
   \subfigure[Uso del GPU desde el menú de selección de nivel.] {\includegraphics[width=0.4 \textwidth]{04ResultadosObetnidos/imagenes/rendimiento02.png}}
   
   \subfigure[Uso del GPU desde una cinemática.] {\includegraphics[width=0.4 \textwidth]{04ResultadosObetnidos/imagenes/rendimiento03.png}}
   
   \subfigure[Uso del GPU desde un nivel de plataforma.] {\includegraphics[width=0.4 \textwidth]{04ResultadosObetnidos/imagenes/rendimiento05.png}}
   
   \subfigure[Uso del GPU desde un nivel de jefe.] {\includegraphics[width=0.4 \textwidth]{04ResultadosObetnidos/imagenes/rendimiento10.png}}
   
  \caption{Resultados del rendimiento del GPU del dispositivo Huawei}
  \label{fig:GPUHuawei}
\end{figure}
                \begin{figure}[h]
                        \centering
                        \includegraphics[width=0.2\textwidth]{04ResultadosObetnidos/imagenes/baterry02.png}
                        \caption{Pantalla de la aplicación \textit{Battery Doctor} para medir el
                        rendimiento del juego.}
                        \label{fig:BateriaYolotl}
                \end{figure}
\subsubsection{Conclusiones de la prueba}
De esta medición del desempeño del GPU se puede observar que la aplicación
utiliza un mínimo del 9\% del GPU y hasta un máximo del casi el 30\%. Por su
parte, el juego utiliza en promedio un 40\% de la batería. Estas cifras son
buenas si se considera que otras aplicaciones como \textit{Messenger} de
\textit{Facebook} llega a utilizar el 50.6\% del GPU y casi el 60\% de la batería
del teléfono, ver figura \ref{fig:BateriaFacebook}.
\begin{figure}[h]
                        \centering
                        \includegraphics[width=0.2\textwidth]{04ResultadosObetnidos/imagenes/baterry01.png}
                        \caption{Pantalla de la aplicación \textit{Battery Doctor} para medir el
                        rendimiento de \textit{Messenger} de \textit{Facebook}.}
                        \label{fig:BateriaFacebook}
                \end{figure}
\subsection{Prueba de aceptación}
Esta prueba está basada en modelo de pruebas \textit{Game Flow}\cite{gameflow}. 
\textit{Game Flow} utiliza estrategias heurísticas de usabilidad y experiencia de usuario.
Este modelo permite medir varios aspectos de los cuales en este proyecto se probarán: 
 {\it joyment} (disfrute), diseño de interfaces, mecánicas y jugabilidad. 
 
Se tomo la decisión de aplicar las pruebas \textit{Game Flow} por nivel ....
 
Esta prueba de aceptación se dividió en dos partes: .... presencial y en línea.


 
\subsubsection{Objetivo} %% debe concordar con la parte de arriba.
Obtener la opinión de los usuarios sobre los elementos de un nivel, tales como
la mecánica, la jugabilidad, las interfaces, los enemigos, etc.

\subsubsection{Herramientas}

Para aplicar las pruebas se utilizaron .... %describir las herramientas
Apk del juego, cuestionario(ver anexo \ref{Anexo:Cuestionario}) y encuesta en \textit{Google Docs}.

\subsubsection{Aplicación}
Para estas pruebas se requieren grupos de personas para probar los niveles del
juego.

%Instrucciones <- crear subsubsection
Para realizar
la prueba se le proporciona al jugador el {\it link} para descargar la apk del juego
y el link de la encuesta. 

La prueba de aceptación está diseñada para ser la más larga, ya que
se busca que el mayor número de personas puedan probar el juego. %% replantearlo

Esta prueba se realiza de dos maneras diferentes:
\begin{itemize}
        \item Publicando los links de la apk y de la encuesta en redes sociales,
        indicando las instrucciones de responder la encuesta por nivel terminado.
        \item Realizando pruebas presenciales a grupos de personas.
\end{itemize}

En el caso de las pruebas presenciales, además de la encuesta se puede observar
las reacciones reales de lo jugadores mientras prueban el juego.

%% separar la aplicación de la conclusión. 
En muchos casos se pudo observar a diferentes grupos de amigos compitiendo por acabar el
nivel, jugadores gritando de alegría al acabar un nivel que les había costado
mucho esfuerzo o exclamaciones llenas de emoción al ser derrotados de último
momento por un jefe (ver figura \ref{fig:AlumnosESCOM}).
\begin{figure}
  \centering
 
   \subfigure[Dos alumnos de la Escuela Superior de Cómputo probando el juego.] {\includegraphics[width=0.4 \textwidth]
   {04ResultadosObetnidos/imagenes/usuarios01}}
        
        \subfigure[Grupo de alumnos de la Escuela Superior de Cómputo probando el
        juego.] {\includegraphics[width=0.4 \textwidth]{04ResultadosObetnidos/imagenes/usuarios02}}
  \caption{Resultados de la herramienta \textit{profiler} al analizar una cinemática.}
  \label{fig:AlumnosESCOM}
\end{figure}
\subsubsection{Conclusiones de la prueba}

A continuación se presentan algunos de los resultados de la encuesta, 
lo resultados completos se encuentran en ..... 
\begin{itemize} %% separar resultados de conclusiones
        \item La principal marca de dispositivos con el que fue probado el juego fue
        Motorola con sistema operativo Android 7.
        \item La mayoría de los usuarios consideran como bueno el movimiento del
        personaje, pero consideran que haciendo más estable el salto el control del
        personaje mejoraría.
        \item La mayoría de los usuarios consideran que la respuesta de la
        \textit{GUI} es buena; sin embargo, recomiendan mejorar el tiempo de respuesta
        de ésta y agregar una animación que indique que un botón ha sido oprimido.
        \item La mayoría de los usuarios opina que la actualización de la barra de
        tonali es buena pero les gustaría que existiera un indicador numérico para ver
        la cantidad de disparos que les queda.
        \item La mayoría de los usuarios considera que lo hace hace débil a un personaje
        es su patrón de movimiento y no la cantidad de daño que pueda generar; por
        otro lado también la mayoría de los usuarios considera que lo que hace a un
        enemigo fuerte es su patrón de movimiento.
        \item El Fantasma morado es considerado por muchos usuarios como el enemigo más
        poderoso en los niveles de plataforma, porque se si se deseara hacer niveles
        más difíciles este debería de ser el enemigo predominante.
        \item El fantasma rojo es considerado por muchos usuarios como el enemigo más
        débil en los niveles de plataforma, porque se si se deseara hacer niveles
        más fáciles este debería de ser el enemigo predominante.
        \item Las dos principales causas de muerte en los jugadores son el tiempo
        de respuesta de la \textit{GUI} y que los enemigos eran demasiado fuertes.
        \item La mayoría de los usuarios consideran sus muertes como un factor de reto
        en el juego. Considerando que la principal causa de muerte fue el tiempo de
        respuesta de la \textit{GUI}, se puede concluir que mejorando este factor se
        disminuiría el porcentaje de jugadores que consideran como factor de estrés su
        muerte.
\end{itemize}

Las conclusiones que se obtuvieron de las pruebas son:

%% sección aparte
Adicionalmente lo jugadores hicieron observaciones y peticiones que ellos
consideran podrían mejorar la experiencia de juego:
\begin{itemize}
        \item Animación que indique que un enemigo ha recibido daño.
        \item Barra de vida para los enemigos.
        \item Posibilidad de que el jugador se agache.
        \item Mensajes de confirmación para los botones que llevan al menú de selección 
        y que cierran la aplicación.
        \item Mejorar el comportamiento de los disparos.
\end{itemize}
Si se desean consultar las gráficas se puede consultar el anexo \ref{Anexo:resultados}.


\section{Niveles terminados}
Para que un nivel pueda ser considerado terminado debe de tener al menos la 
funcionalidad especificada en el documento de diseño, la cual contempla lo 
siguiente: 
\begin{itemize}
	\item Actualización de la barra de vida.
	\item Actualización de la barra de \textit{tonalli}
	\item Actualización de los marcadores, en caso de que el nivel contenga objetos 
	coleccionables.
	\item Enemigos, salvo por el primer nivel.
	\item Obstaculos.
	\item Ítems, salvo por el primer nivel.
	\item Control del personaje por medio de la \textit{GUI}.
	\item Gestion de la muerte del jugador.
	\item Puntos de guardado.
\end{itemize} 

En la figura \ref{fig:NivelesPares} se muestra la funcionalidad con la que cuentan los niveles pares.
		
		\begin{figure}[h]
    			\centering
    			\includegraphics[width=0.9\textwidth]{04ResultadosObetnidos/imagenes/funcionalidadPares.png}
    			\caption{Funcionalidad con la que cuentan los niveles pares, la P es 
    			para los niveles de plataforma y la J para los niveles de jefes.}
    			\label{fig:NivelesPares}
		\end{figure}