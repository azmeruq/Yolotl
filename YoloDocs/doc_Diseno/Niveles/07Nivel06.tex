	\section{Nivel 6} \label{Nivel:Niv06}
	\subsection{Título del nivel}
	Sin gravedad.
	\subsection{Encuentro}
	Este nivel estará disponible después de vencer al jefe del quinto nivel (ver apartado \ref{Nivel:Niv05}).
	\subsection{Descripción}
	A la entrada del Pancuetlacalóyan, Xólotl le advierte a Malinalli sobre los peligros que los esperan en los dominios de Tlazoltéotl. Para ayudar a Malinalli, Xólotl adopta la forma de un ave. Juntos surcaran los cielos del Pancuetlacalóyan con el objetivo de reclamar el poder de Tlazoltéotl.
	\subsection{Objetivos}
	En este nivel el jugador:
	\begin{itemize}
		\item Atravesar la zona de plataformas del nivel. En este nivel Xólotl (ver aparatado \ref{per:xolotl}) se transformara en un ave y surcara el nivel volando por lo que el jugador deberá de mantenerse sobre él para poder avanzar. Xólotl contara con una velocidad predefinida y el jugador no la podrá modificar, Xólotl sólo disminuirá su velocidad en las zonas con enemigos para darle tiempo al jugador de poder centrarse en la batalla o en las zonas de plataformas para que el jugador pueda obtener ítems. Xólotl contará con una barra de vida parecida a la de Malinalli, esta barra estará ubicada en la parte superior derecha de la pantalla y estará precedida por la imagen de Xólotl. La cantidad de vida de Xólotl disminuirá si colisiona con  los enemigos o con sus ataques; si la barra de vida de Xólotl llega cero, éste desaparecerá y el jugador perderá la partida. Al igual que Malinalli, Xólotl podrá recuperar vida con el ítem grano de cacao (ver apartado \ref{item:cacao}).
		\item Derrotar a Tlazoltéotl. Tlazoltéotl aparecerá volando y Xólotl la seguirá. El jugador deberá evitar los ataques de la diosa sin caerse de Xólotl. Tlazoltéotl estará rodeada de un circulo de tonalli corrupto, por lo que el jugador deberá disparar tonalli para destruir el circulo que protege a Tlazoltéotl. Tlazoltéotl será inmune a cualquier ataque mientras se encuentre activo el circulo de tonalli corrupto; una vez destruido el circulo de tonalli corrupto, los ataques del jugador podrán herir a Tlazoltéotl. Tlazoltéotl estará sin el circulo de tonalli corrupto por un tiempo y después lo volverá a invocar. 
	\end{itemize}	 
	\subsection{Progreso}
	 Al terminar el nivel el jugador:
\begin{itemize}
        \item Habrá incrementado la cantidad de vida de Malinalli. 
        \item Desbloqueara las siguientes cinemáticas:
\begin{itemize}
        \item Cinemática 29 (ver apartado \ref{Cin:Cinematica29}). 
        \item Cinemática 30 (ver apartado \ref{Cin:Cinematica30}).
        \item Cinemática 31 (ver apartado \ref{Cin:Cinematica31}).
\end{itemize}
        \item Desbloqueará El nivel 7 (ver apartado  \ref{Nivel:Niv07}) del juego en el menú seleccionable (ver apartado \ref{inter:interfaz03}).
\end{itemize}
	\subsection{Enemigos}
	\begin{itemize}
		\item Fantasmas rojos (ver apartado \ref{per:fantasmaR}).
		\item Zopilote	(ver apartado \ref{per:zopilote}).
		\item Tlazoltéotl (ver apartado \ref{per:tlazolteotl}).
	\end{itemize}
	\subsection{Items}
\begin{itemize}
        \item   Cacao (ver apartado \ref{item:cacao}).
        \item Flor de Vainilla (ver apartado \ref{item:vainilla}).
\end{itemize}
	\subsection{Personajes}
	\begin{itemize}
		\item Malinalli (ver apartado \ref{per:malinalli}).
			
		\item Xolotl (ver apartado\ref{per:xolotl}).
		
		\item Tlazoltéotl (ver apartado \ref{per:tlazolteotl}).
			
	\end{itemize}
\subsection{Escenario}
\begin{itemize} 
	\item Fondo: El cielo de este nivel es de color rojo. Las nubes se observan a la lejanía formando un espiral y bajo ellas se pueden observar varios truenos que caen sobre montañas.
	\item Suelo: Rocoso obscuro.
	\item Obstáculos: Caer al vacío.
	\item Viento en contra. Ver en \ref{obs.vientoM}.
	\item Objetos de fondo: Sin objetos de fondo
\end{itemize}	
	\subsection{Referencia a BGM y SFX}
	\begin{itemize}
		\item BGM
			\begin{itemize}
				\item Música plataforma sexto nivel (ver apartado \ref{Musica:N06_ZN01}).
				\item Música jefe sexto nivel (ver apartado \ref{Musica:N06_ZN02}).
			\end{itemize}
		\item SFX
		\begin{itemize}
			\item Risa de mujer (ver apartado \ref{SFX:risaM}).
			\item Viento (ver apartado \ref{SFX:Viento}).
			\item Liquido viscoso (ver apartado \ref{SFX:ligVisc}).
		\end{itemize}
	\end{itemize}
	\subsection{Referencia a FX}
	\begin{itemize}
		\item Explosión de Tonalli corrupto (ver apartado \ref{FX:ExTonCor}).
		\item Explosión de energía tonalli rojo (ver apartado \ref{FX:ExpTonR})
	\item Explosión de energía tonalli verde (ver apartado \ref{FX:ExpTonV})
	\item Relámpagos (ver apartado \ref{FX:Relam}).
	\end{itemize}
	