\section{Nombre: Fantasma morado}   \label{per:fantasmaM}
\subsection{Descripción:}
Este enemigo nace de las almas que no quedaron varadas en los niveles del Mictlán al no poder terminar los desafíos. Tiene la forma de un cráneo, en los orificios de los ojos del cráneo se puede ver un brillo espectral. El cráneo está decorado con diferentes patrones de color morado y verde en la zona superior del cráneo, la frente, los ojos y la boca. De la parte trasera del cráneo sale fuego cuya paleta de colores es un degradado de morado oscuro y guinda, siendo el morado oscuro el predominante.  
\subsection{Status:}
Enemigo.
\subsection{Imagen}
Ver figura \ref{fig:fantasmaM}.
\begin{figure}
	\centering
	\includegraphics[height=0.2 \textheight]{Imagenes/fantasmaMorado}
	\caption{Fantasma morado.}
	\label{fig:fantasmaM}
\end{figure}
\subsection{Encuentro:}
En niveles 2, 4, 5, 7, 9.
\subsection{Habilidades:}
\begin{itemize}
	\item Embestida. Ver habilidad \ref{hab.embestida}.
	
\end{itemize}
\subsection{Patrón de ataque:}
El enemigo realizara un recorrido vertical del punto A al B y del B al A de manera cíclica mientras ejecuta la habilidad de disparo rojo.
\subsection{Bloques de animación:}
	\begin{itemize}
		\item Animación fuego.
	\end{itemize}