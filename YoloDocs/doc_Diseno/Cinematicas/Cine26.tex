\section{Cinemática 26. Guarida de Tepeyóllotl en el Teyollocualoyan. int/día.}  \label{Cin:Cinematica26}
 \textsc{Personajes}:
 \begin{itemize}
 \item Tepeyóllotl (Sin coraza).
\item Mictecacíhuatl
 \end{itemize}
\textit{Tepeyóllotl está acostado en el centro de la sala sobre telas finas de algodón. La decoración del lugar es ostentosa con diferentes objetos de oro. Al lado de Tepeyóllotl hay un espejo de jade. La figura de   Mictecacíhuatl se proyecta del espejo.}
\begin{center}
\textsc{\underline{Mictlantecuhtli}}
\\
\par
\textsc{\underline{Itzpápalotl}} y Mictlecayotl han caído. El invasor sigue avanzando. Era tu deber detenerlo, en lugar de eso estas dormitando.
\\
\par
\textsc{\underline{Tepeyóllotl}}
\\
\par
Dilo directamente, me culpas por el avance de Xólotl.
\\
\par
\textsc{\underline{Mictlantecuhtli}}
\\
\par
Es el deber de un guardián detener a su enemigo y en caso de que se vea superado por éste, debe de morir llevándose a su enemigo consigo. Huir no está permitido. Hemos perdido tres de los nueve guardianes y cada guardián caído es equivalente a más poder para Xólotl.
\\
\par
\textsc{\underline{Tepeyóllotl}}
\\
\par
No creo que Xólotl sea el problema.
\\
\par
\textsc{\underline{Mictlantecuhtli}}
\\
\par
Explícate.
\\
\par
\textsc{\underline{Tepeyóllotl}}
\\
\par
Xólotl es acompañado por un mortal, es ella la que se enfrenta a los Dioses. Xólotl no tiene la fuerza ni la valentía para hacerlo, pero si tiene la mente para planearlo. Me pregunto si la niña mortal es solo valentía y fuerza o, por el contrario.
\\
\par
\textsc{\underline{Mictlantecuhtli}}
\\
\par
Una mortal dispuesta a asesinar dioses. Jamás había oído algo como eso. ¿Cómo puede existir algo así?
\\
\par
 \textsc{\underline{Tepeyóllotl}}
\\
\par
Una esclava mortal. Xólotl debió de haberle ofrecido algo. Los humanos son criaturas simples, rara vez se embarcan en peligrosas empresas si no van a obtener algo que amerite el esfuerzo. Me enfrentare a la mortal una vez más. Piensa en lo que te dije, será de utilidad si fallamos.
\end{center}