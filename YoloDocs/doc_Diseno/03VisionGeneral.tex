\chapter{Visión general del juego}
Yolotl nace a partir del deseo de contar una historia con la capacidad de tocar los sentimientos del jugador, por tal motivo el juego está más orientado a la narrativa que a la jugabilidad. Sin embargo, esto no es sinónimo de que la jugabilidad no es un factor importante en el desarrollo del juego sino todo lo contrario: es otro vehículo de la narrativa permitiendo al jugador encarnar de mejor manera a los personajes al hacerlo participe de la narrativa.
\\
\par
La visión narrativa que acompaña a Yolotl es tal que su argumento combina mitología y sucesos históricos para darle una capa extra de profundidad narrativa. Permitiendo que el juego no solo sea un medio de diversión sino a su vez ofrezca una visión histórica y cultural del México prehispánico. 
\\
\par
 Una de las fortalezas de Yolotl son sus personajes, cada personaje fue cuidadosamente diseñado, prestando principal atención en los pequeños detalles que muestran los aspectos de sus personalidades. El peso que tienen los personajes sobre el juego es tal que el diseño de los niveles del Mictlán son un reflejo de su guardián, este cuidado de diseño abarca desde los enemigos comunes a enfrentar hasta la musicalización del nivel, culminando en los bloques de animación del jefe del nivel cuando se le enfrenta.
\\
\par
Desafortunadamente, esta visión narrativa del juego no es gratuita y la jugabilidad se ve afectada al ofrecer una progresión lineal al jugador, en donde cada nivel tiene objetivos muy fijos y la capacidad de exploración se ve bastante limitada. No obstante, la composición de clases de programación del juego permitirá que se puedan agregar nuevas funcionalidades sin afectar las clases y el código ya existente en un futuro.
\\
\par
Yolotl es un juego para el público que disfrute de juegos cuyo punto fuerte sea la narrativa por sobre los juego de mundo abierto. 
